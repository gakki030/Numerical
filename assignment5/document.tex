\documentclass[UTF8]{ctexart}
\usepackage{listings}
\usepackage{textcomp} % 必须加上,否则报错
\usepackage{xcolor}
\usepackage{amsmath}
\pagestyle{plain}
\usepackage{fontspec}
\usepackage{graphicx}  
\usepackage{epstopdf}
\CTEXsetup[format={\Large\bfseries}]{section}
\lstset{language=Matlab}%代码语言使用的是matlab
%\lstset{breaklines}%自动将长的代码行换行排版
\lstset{extendedchars=false}%解决代码跨页时,章节标题,页眉等汉字不显示的问题
\CTEXoptions[today=old] 
\title{Assignment 5}  %文章标题
\author{Zou Yuan\\Student No. 21960216}   %作者的名称
\date{\today}       %日期
\begin{document}
	\maketitle
	\section{P199, Computer Problems:1.}
	Form the normal equations, and compute the least squares solution and 2-norm error for the inconsistent systems.

	The Matlab code is as follows:
	
	\begin{centering}
	\begin{lstlisting}
	
	A=[3,-1 2;4 1 0;-3 2 1;1 1 5;-2 0 3];
	b=[10;10;-5;15;0]; 
	x1=GaussElimination(A'*A,(A'*b)')
	r1=norm(b-A*x1,2)
	B=[4 2 3 0;-2 3 -1 1;1 3 -4 2;1 0 1 -1;3 1 3 -2];
	c=[10;0;2;0;5];
	x2=GaussElimination(B'*B,(B'*c)')
	r2=norm(c-B*x2,2)
	
	function [r_matrix] = GaussElimination(a,b)
	%  inputs:
	%         a:系数矩阵,为n*n维方阵
	%         b:载荷矩阵,为n*1维矩阵
	%  outputs:
	%         r_matrix:计算结果向量,为n*1为矩阵
	
	% 判断输入矩阵维度是否满足要求
	[row_coeff,col_coeff] = size(a);
	[row_load,~] = size(b');
	% 初始化r_matrix矩阵
	r_matrix = zeros(row_load,1);
	% 判断输入的维度是否满足要求
	if (row_coeff ~= col_coeff) || (row_coeff ~= row_load)
	% 不满足则输出错误提示
	print('输入错误!');
	%消去过程
	else
	n=row_coeff;
	for j = 1 : n-1
	if abs(a(j,j))<eps; error('zero pivot encountered'); 
	end
	for i = j+1 : n
	mult = a(i,j)/a(j,j);
	for k = j+1:n
	a(i,k) = a(i,k) - mult*a(j,k);
	end
	b(i) = b(i) - mult*b(j);
	end
	end
	%回代过程
	r_matrix(n) = b(n)/a(n,n);
	for k = n-1:-1:1
	sum_temp = 0;
	for j = k+1:n
	sum_temp = sum_temp + a(k,j)*r_matrix(j);
	end
	r_matrix(k) = (b(k) - sum_temp)/a(k,k);
	end
	end % 条件判断结束
	end
	\end{lstlisting}
		\end{centering}
The input is the system matrix,where the least squares solution is:
\begin{align*}
&\begin{bmatrix}x_1\\x_2\\x_3\end{bmatrix}=\begin{bmatrix}
2.5246    \\ 0.6616   \\2.0934
\end{bmatrix}
&\begin{bmatrix}x_1\\x_2\\x_3\\x_4\end{bmatrix}=\begin{bmatrix}
1.2739\\ 0.6885 \\1.2124\\1.7497
\end{bmatrix}
\end{align*}
the 2-norm error is:$$r1 =2.413 \qquad r2 =0.8256$$
\section{P225, Computer Problems:1.}
Write a Matlab program that implements classical Gram–Schmidt to find the reduced QR
factorization. Check your work by comparing factorizations of the matrices in Exercise 1 with
the Matlab qr(A,0) command or equivalent. The factorization is unique up to signs of the
entries of Q and R.
   \begin{centering}
   	\begin{lstlisting}
   	A=[4,0;3,1];
   	B=[1,2;1,1];
   	C=[2,1;1,-1;2,1];
   	D=[4,8,1;0,2,-2;3,6,7];
   	[q1,r1]=Schmidt(A)
   	[q11,r12]=qr(A,0)
   	
   	[q2,r2]=Schmidt(B)
   	[q22,r22]=qr(B,0)
   	
   	[q3,r3]=Schmidt(C)
   	[q33,r33]=qr(C,0)
   	
   	[q4,r4]=Schmidt(D)
   	[q44,r44]=qr(D,0)
   	
   	function [Q,R]=Schmidt(A)
   	[row,col] = size(A);  
   	R=zeros(row,col);
   	Q=zeros(row,row);
   	for j=1:col
   	y=A(:,j);
   	for i=1:j-1
   	R(i,j)=(Q(:,i))'*A(:,j);
   	y=y-R(i,j)*Q(:,i);
   	end
   	R(j,j)=norm(y,2);
   	Q(:,j)=y/R(j,j);
   	end
   	end
   	
   	\end{lstlisting}   \end{centering}
To compare the outputs between Schmidt and the Matlab qr(A,0) command:
$$q1=\begin{bmatrix}
	
	0.8000 &  -0.6000\\
	0.6000 &  0.8000
\end{bmatrix}\qquad
r1= \begin{bmatrix}
 5.0000  &  0.6000\\
     0   & 0.8000
\end{bmatrix} $$
$$q11 =\begin{bmatrix}
-0.8000  & -0.6000\\
-0.6000  &  0.8000
\end{bmatrix}\qquad
r11=\begin{bmatrix}	
  -5.0000 &  -0.6000\\
0  &  0.8000
\end{bmatrix}$$

$$q2=\begin{bmatrix}

0.7071  &  0.7071\\
0.7071  & -0.7071
\end{bmatrix}\qquad
r2= \begin{bmatrix}
 1.4142  &  2.1213\\
0   & 0.7071
\end{bmatrix} $$
$$q22 =\begin{bmatrix}
-0.7071  & -0.7071\\
-0.7071   & 0.7071

\end{bmatrix}\qquad
r22=\begin{bmatrix}	
-1.4142 &  -2.1213\\
0   &-0.7071
\end{bmatrix}$$    
   
$$q3=\begin{bmatrix}
	
0.6667  &  0.2357     &    0\\
0.3333  & -0.9428      &   0\\
0.6667  &  0.2357     &    0

\end{bmatrix}\qquad
r3= \begin{bmatrix}
	 3.0000   & 1.0000\\
	0   & 1.4142\\
	0    &     0
	
\end{bmatrix} $$
$$q33 =\begin{bmatrix}
-0.6667  &  0.2357\\
-0.3333   &-0.9428\\
-0.6667  &  0.2357
	
\end{bmatrix}\qquad
r33=\begin{bmatrix}	
	 -3.0000  & -1.0000\\
	0   & 1.4142
\end{bmatrix}$$    

$$q4=\begin{bmatrix}
	
  0.8000     &    0  & -0.6000\\
0   & 1.0000    &     0\\
0.6000     &    0  &  0.8000
	
\end{bmatrix}\qquad
r4= \begin{bmatrix}
5  &  10   &  5\\
0   &  2  &  -2\\
0   &  0  &   5\\
	
\end{bmatrix} $$
$$q44 =\begin{bmatrix}
	-0.8000      &   0 &  -0.6000\\
	0  &  1.0000     &    0\\
	-0.6000   &      0 &   0.8000	
\end{bmatrix}\qquad
r44=\begin{bmatrix}	
 -5  & -10   & -5\\
0   &  2   & -2\\
0   &  0  &   5
\end{bmatrix}$$    


\section{P225, Computer Problems:3.}
Repeat Computer Problem 1, but implement Householder reflections.
Code is below:
   \begin{centering}
	\begin{lstlisting}	
	
	A=[4 0;3 1];
	B=[1 2;1 1];
	C=[2 1;1 -1;2 1];
	D=[4 8 1;0 2 -2;3 6 7];
	[Q1,R1]=qr(A)
	[q1,r1]=qrhs(A)
	[Q2,R2]=qr(B)
	[q2,r2]=qrhs(B)
	[Q3,R3]=qr(C)
	[q3,r3]=qrhs(C)
	[Q4,R4]=qr(D)
	[q4,r4]=qrhs(D)
	function [H,rho]=householder(x,y)
	
	x=x(:);
	y=y(:);
	if length(x)~=length(y)
	error('Error!');
	end
	rho=-sign(x(1))*norm(x)/norm(y);
	y=rho*y;
	v=(x-y)/norm(x-y);
	I=eye(length(x));
	H=I-2*v*v';
	end
	
	function [Q,R]=qrhs(A)   	
	n=size(A,1);   	
	R=A;   	
	Q=eye(n);   	
	for i=1:n-1   	
	x=R(i:n,i);	
	y=[1;zeros(n-i,1)];  	
	Ht=householder(x,y);   	
	H=blkdiag(eye(i-1),Ht);   	
	Q=Q*H;   	
	R=H*R;   	
	end
	end
	
	\end{lstlisting}   \end{centering}
The output is:
$$Q1=\begin{bmatrix}
-0.8000  & -0.6000\\
-0.6000  &  0.8000
\end{bmatrix}\qquad
R1= \begin{bmatrix}
-5.0000  & -0.6000\\
0   & 0.8000
\end{bmatrix} $$
$$q1 =\begin{bmatrix}
-0.8000  & -0.6000\\
-0.6000  &  0.8000
\end{bmatrix}\qquad
r1=\begin{bmatrix}	
-0.8000  & -0.6000\\
-0.6000  &  0.8000
\end{bmatrix}$$

$$Q2=\begin{bmatrix}

-0.7071  &  -0.7071\\
-0.7071  & 0.7071
\end{bmatrix}\qquad
R2= \begin{bmatrix}
-1.4142  &  -2.1213\\
0   & -0.7071
\end{bmatrix} $$
$$q2 =\begin{bmatrix}
-0.7071  & -0.7071\\
-0.7071   & 0.7071

\end{bmatrix}\qquad
r2=\begin{bmatrix}	
-1.4142 &  -2.1213\\
0   &-0.7071
\end{bmatrix}$$    

$$Q3=\begin{bmatrix}

   -0.6667   & 0.2357 &  -0.7071\\
-0.3333 &  -0.9428  & -0.0000\\
-0.6667   & 0.2357 &   0.7071

\end{bmatrix}\qquad
R3= \begin{bmatrix}
-3.0000   &-1.0000\\
0   & 1.4142\\
0    &     0
\end{bmatrix} $$
$$q3 =\begin{bmatrix}
   -0.6667   & 0.2357 &  -0.7071\\
-0.3333 &  -0.9428  & -0.0000\\
-0.6667   & 0.2357 &   0.7071

\end{bmatrix}\qquad
r3=\begin{bmatrix}	
-3.0000   &-1.0000\\
0   & 1.4142\\
0    &     0
\end{bmatrix}$$    

$$Q4=\begin{bmatrix}

-0.8000     &    0  & -0.6000\\
0   & 1.0000    &     0\\
-0.6000     &    0  &  0.8000

\end{bmatrix}\qquad
r4= \begin{bmatrix}
-5  &  -10   & -5\\
0   &  2  &  -2\\
0   &  0  &   5\\

\end{bmatrix} $$
$$q4 =\begin{bmatrix}
-0.8000      &   0 &  -0.6000\\
0  &  1.0000     &    0\\
-0.6000   &      0 &   0.8000	
\end{bmatrix}\qquad
r4=\begin{bmatrix}	
-5  & -10   & -5\\
0   &  2   & -2\\
0   &  0  &   5
\end{bmatrix}$$    
\end{document}
