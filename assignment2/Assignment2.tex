\documentclass[UTF8]{ctexart}
\usepackage{listings}
\usepackage{textcomp} % 必须加上,否则报错
\usepackage{xcolor}
\usepackage{amsmath}
\pagestyle{plain}
\usepackage{fontspec}
\CTEXsetup[format={\Large\bfseries}]{section}
\lstset{language=Matlab}%代码语言使用的是matlab
\lstset{breaklines}%自动将长的代码行换行排版
\lstset{extendedchars=false}%解决代码跨页时,章节标题,页眉等汉字不显示的问题
\CTEXoptions[today=old] 
\title{Assignment 2}  %文章标题
\author{Zou Yuan\\Student No. 21960216}   %作者的名称
\date{\today}       %日期
\begin{document}
	\maketitle
	\section{Gauss Elimination}
	The problems to be solved in this section are below:
	
	Put together the code fragments in this section to create a Matlab program for “naive"
	Gaussian elimination (meaning no row exchanges allowed). Use it to solve the systems of
	Exercise 2:
	\begin{align}
	\begin{split}
	2x − 2y − z &=−2\\
	x + y − 2z &= 1\\
	−2x + y − z &=−3\\
	\end{split}
	\end{align}
	\begin{align}
	\begin{split}
	x + 2y − z &= 2\\
	3y + z &= 4\\
	2x − y + z &= 2\\
	\end{split}
	\end{align}
	\begin{align}
	\begin{split}
	2x + y − 4z &=−7\\
	x − y + z &=−2\\
	−x + 3y − 2z &= 6\\
	\end{split}
	\end{align}
	
	The Matlab code is as follows:
	
	\begin{centering}
	\begin{lstlisting}
	function [r_matrix] = GaussElimination(a,b)
	%   a:系数矩阵,为n*n维方阵
	%   b:输出向量,为n*1维矩阵
	%   r_matrix:计算结果向量,为n*1为矩阵
	
	% 判断输入矩阵维度是否满足要求
	[row_coeff,col_coeff] = size(a);
	[row_load,~] = size(b');
	% 初始化r_matrix矩阵
	r_matrix = zeros(row_load,1);
	% 判断输入的维度是否满足要求
	if (row_coeff ~= col_coeff) || (row_coeff ~= row_load)
	% 不满足则输出错误提示
	print('输入错误!');
	%消去过程
	else
	n=row_coeff;
	for j = 1 : n-1
	if abs(a(j,j))<eps; error('zero pivot encountered'); 
	end
	for i = j+1 : n
	mult = a(i,j)/a(j,j);
	for k = j+1:n
	a(i,k) = a(i,k) - mult*a(j,k);
	end
	b(i) = b(i) - mult*b(j);
	end
	end
	%回代过程
	r_matrix(n) = b(n)/a(n,n);
	for k = n-1:-1:1
	sum_temp = 0;
	for j = k+1:n
	sum_temp = sum_temp + a(k,j)*r_matrix(j);
	end
	r_matrix(k) = (b(k) - sum_temp)/a(k,k);
	end
	end % 条件判断结束
	end
	\end{lstlisting}
		\end{centering}

The inputs are the coefficient matrixs and output vectors of the following expression:
   $$ \begin{bmatrix} 
	2  &-2  &-1 \\4 &1  &-2 \\-2  &1 &-1
    \end{bmatrix}\times
    \begin{bmatrix}x_1\\x_2\\x_3\end{bmatrix}=\begin{bmatrix}
     -2     \\1    \\-3
    \end{bmatrix}
    \eqno(1)$$
    
    $$ \begin{bmatrix} 
    1  &2  &-1 \\0&3  &1 \\2  &-1 &1
    \end{bmatrix}\times
    \begin{bmatrix}x_1\\x_2\\x_3\end{bmatrix}=\begin{bmatrix}
    2     \\4    \\2
    \end{bmatrix}
    \eqno(2)$$
    
    $$ \begin{bmatrix} 
    2  &1  &-4 \\1 &-1  &1 \\-1  &3 &-2
    \end{bmatrix}\times
    \begin{bmatrix}x_1\\x_2\\x_3\end{bmatrix}=\begin{bmatrix}
    -7     \\-2   \\6
    \end{bmatrix}
    \eqno(3)$$
    
    The output are as below:
   $$ \begin{bmatrix}x_1\\x_2\\x_3\end{bmatrix}=\begin{bmatrix}
    	1     \\1    \\2
    \end{bmatrix}
    \eqno(1)$$
    $$ \begin{bmatrix}x_1\\x_2\\x_3\end{bmatrix}=\begin{bmatrix}
    1     \\1    \\1
    \end{bmatrix}
    \eqno(2)$$
    $$ \begin{bmatrix}x_1\\x_2\\x_3\end{bmatrix}=\begin{bmatrix}
    -1     \\3   \\2
    \end{bmatrix}
    \eqno(3)$$
   \section{LU Factorization}
   The problems to be solved in this section are below:
   
   Use the code fragments for Gaussian elimination in the previous section to write a Matlab
   script to take a matrix A as input and output L and U. No row exchanges are allowed—the
   program should be designed to shut down if it encounters a zero pivot. Check your program by
   factoring the matrices in Gauss Elimination.
   
   The Matlab code is as follows:
   
   \begin{centering}
   	\begin{lstlisting}
   	function [L_matrix,U_matrix] = LU(a)
   	% 判断输入矩阵维度是否满足要求
   	[row_coeff,col_coeff] = size(a);
   	if (row_coeff ~= col_coeff) 
   	error('输入错误!'); 
   	end
   	% 初始化L_matrix矩阵
   	n=row_coeff;
   	L_matrix = zeros(n,n);            
   	for i = 1:n;     L_matrix(i,i) = 1;  end   
   	for j = 1 : n-1
   	if abs(a(j,j))<eps
   	error('zero pivot encountered');   
   	end
   	for i = j+1 : n
   	mult = a(i,j)/a(j,j);
   	L_matrix(i,j) = mult;  
   	for k = j:n
   	a(i,k) = a(i,k) - mult*a(j,k);
   	end
   	end
   	U_matrix=a;
   	end
   	
   	\end{lstlisting}
   \end{centering}

The output is:
\begin{equation*}L = \begin{bmatrix}1 & 0 & 0\\2 &1&0\\1& 0 & 1\end{bmatrix}\qquad U =\begin{bmatrix}3 & 1 & 2\\0 &1&0\\0& 0 & 3\end{bmatrix}\eqno(1)
\end{equation*}
\begin{equation*}L = \begin{bmatrix}1.0000 & 0 & 0\\1.0000 &1.0000&0\\0.5000& 0.5000 & 1.0000\end{bmatrix}\qquad U =\begin{bmatrix}4 & 2 & 0\\0 &2&2\\0& 0 & 2\end{bmatrix}\eqno(2)
\end{equation*}

\begin{equation*}L = \begin{bmatrix}1 & 0 & 0&0\\0 &1&0&0\\1& 2 & 1&0\\0&1&0&1\end{bmatrix}\qquad U =\begin{bmatrix}1 & -1 & 1&2\\0 &2&1&0\\0& 0 & 1&2\\0&0&0&-1\end{bmatrix}\eqno(3)
\end{equation*}
    
    
\section{SOURCES Of ERROR}
 The problem to be solved in this section is shown below:
For the$ n \times n$ matrix with entries $A_{ij} = 5/(i + 2j − 1)$, set $x = [1, . . . ,1]^T$ and $b = Ax$. Use
the Matlab program from Computer Problem 2.1.1 or Matlab’s backslash command to compute $x_c$, the double precision computed solution. Find the infinity norm of the forward
error and the error magnification factor of the problem $Ax = b$, and compare it with the
condition number of $A: (a) n = 6 (b) n = 10.$

The Matlab code is as follows:

\begin{centering}
	\begin{lstlisting}
	function [ ] = Matrix(n)
	matrix_A=zeros(n);
	for i=1:n
	for k=1:n
	matrix_A(i,k)=5/(i+2*k-1);
	end
	end
	x = ones(1,n);
	b = matrix_A*(x');
	xc = matrix_A\(x');
	r = b-matrix_A*xc;
	back_error=norm(r,inf)/norm(b,inf);
	forward_error=norm(x-xc,inf)/norm(x,inf);
	cond=forward_error/back_error;
	fprintf('back_error=%d',back_error)
	fprintf('forward_error=%d',forward_error)
	fprintf('cond=%d',cond)
	
	\end{lstlisting}
\end{centering}
The $n=6$ output is:
\begin{align*}
back\underline{\hspace{0.5em}}error&=8.367347e-01\\
forward \underline{\hspace{0.5em}} error&=5.636625e+03\\
cond&=6.736454e+03
\end{align*}
The $n=10$ output is:
\begin{align*}
back\underline{\hspace{0.5em}} error&=8.634331e-01\\
forward \underline{\hspace{0.5em}} error&=4.597081e+06\\
cond&=5.324188e+06\end{align*}
\section{PA=LU}
The problem to be solved in this section is to find the PA=LU factorization (using partial pivoting) of the following matrices:
$$
\begin{gathered}
\begin{bmatrix}1 & 1 & 0\\2 &1&-1\\-1& 1 & -1
\end{bmatrix}
\begin{bmatrix}0 & 1 &3\\2 &1&1\\-1& -1& 2
\end{bmatrix}
\begin{bmatrix}1 & 2 & -3\\2 &4&2\\-1& 0 & 3
\end{bmatrix}
\begin{bmatrix}0 & 1 &0\\1 &0&2\\-2& 1& 0
\end{bmatrix}
\end{gathered}
$$

The matlab script is:
\begin{centering}
	\begin{lstlisting}
	A=[1,1,0;2,1,-1;-1,1,-1];
	B=[0,1,3;2,1,1;-1,-1,2];
	C=[1,2,-3;2,4,2;-1,0,3];
	D=[0,1,0;1,0,2;-2,1,0];
	[L,U,P] =lu(A)
	[L,U,P]=lu(B)
	[L,U,P]=lu(C)
	[L,U,P]=lu(D)
		\end{lstlisting}
\end{centering}

The outputs are:
\begin{equation*}L = \begin{bmatrix} 1.0000 &   0  & 0\\-0.5000    &1.0000        & 0\\0.5000  &  0.3333    &1.0000\end{bmatrix} U =\begin{bmatrix}2.0000 &   1.0000   &-1.0000\\   0  & 1.5000  & -1.5000\\ 0 &        0   & 1.0000\end{bmatrix}\quad P=\begin{bmatrix} 0   &  1     &0\\ 0    & 0   &  1\\ 1    & 0  &   0
\end{bmatrix}
\end{equation*}

\begin{equation*}L = \begin{bmatrix} 1.0000 &0  & 0\\    0  &  1.0000      &   0\\-0.5000  &  -0.5000    &1.0000\end{bmatrix}\quad U =\begin{bmatrix} 2  &   1     &1\\    0   &  1   &  3\\      0  &   0     &4\end{bmatrix}\quad P=\begin{bmatrix} 0   &  1     &0\\ 1    & 0   & 0\\ 0   & 0  &  1
\end{bmatrix}
\end{equation*}

\begin{equation*}L = \begin{bmatrix} 1.0000 &  0  & 0\\   -0.5000    &1.0000        & 0\\-0.5000  &  0    &1.0000\end{bmatrix}\quad U =\begin{bmatrix} 2  &   4     &2\\    0   &  2   &  4\\      0  &   0     &-4\end{bmatrix}\quad P=\begin{bmatrix} 0   &  1     &0\\ 0    & 0   & 1\\ 1  & 0  &  0
\end{bmatrix}
\end{equation*}
\begin{equation*}L = \begin{bmatrix} 1.0000 &   0  & 0\\   0    &1.0000        & 0\\-0.5000  &  -0.5000    &1.0000\end{bmatrix} U =\begin{bmatrix} -2  &   1     &0\\    0     &1   &  0\\     0   &  0     &2\end{bmatrix}\quad P=\begin{bmatrix} 0     &0  &   1\\ 1    & 0   & 0\\ 0 & 1  &  0
\end{bmatrix}
\end{equation*}


\end{document}